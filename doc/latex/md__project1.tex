Our first project serves as an introduction to Fortran 95, the role of modules, and initializing data through user-\/defined data structures.

An excellent example of a data module is the {\ttfamily atoms\+\_\+par} module in the {\ttfamily dbsr\+\_\+zcom.\+f90} library for (\href{https://github.com/compas/DBSR_HF}{\texttt{ D\+B\+S\+R\+\_\+\+HF}}), a compas repository. The first few lines of this module are shown here. \begin{DoxyVerb}!======================================================================  
      Module atoms_par    
!======================================================================  
!     atomic parameters according periodic table  
!----------------------------------------------------------------------  
      Implicit none  
      Integer, parameter :: n_atoms = 104    
      Type atomic  
        INTEGER :: an  
        CHARACTER(2) :: symbol  
        CHARACTER(4) :: core
        CHARACTER(40) :: conf
        REAL(8) :: weight
      End type atomic

  Type(atomic), dimension(n_atoms), parameter, public :: atoms = (/  &
   atomic(   1,  'H ',  '    ',  '1s(1)',                   1.0079), &
   atomic(   2,  'He',  '    ',  '1s(2)',                   4.0026), &
   atomic(   3,  'Li',  '[He]',  '2s(1)',                   6.9410), &
   atomic(   4,  'Be',  '[He]',  '2s(2)',                   9.0122), &
   atomic(   5,  'B ',  '[Be]',  '2p-(1)',                 10.8110), &
   atomic(   6,  'C ',  '[Be]',  '2p-(1)2p(1)',            12.0107), &
   atomic(   7,  'N ',  '[Be]',  '2p-(1)2p(2)',            14.0067), &
   atomic(   8,  'O ',  '[Be]',  '2p-(1)2p(3)',            15.9994), &
   etc.
\end{DoxyVerb}
 So {\ttfamily atoms} is an array whose elements are of type {\ttfamily atomic} and the module initializes this array.

In {\ttfamily G\+R\+A\+SP} the finite nucleus is modelled in terms of parameters, namely the atomic number Z, and a mass number, A, although a user may override the values. An important parameter in nuclear physics is the \char`\"{}root-\/mean-\/square\char`\"{} (R\+R\+MS) of the radius. For Z $<$ 91, W.\+R. Johnson, G.\+Soff, Atomic Data and Nuclear Data Tables \{\textbackslash{}bf 33\}, p.\+405 (1985) showed that a good approximation was the formula {\ttfamily R\+R\+MS = 0.\+836 A$^\wedge$(1/3) + 0.\+570} and this is the default in G\+R\+A\+SP, but for many combinations of (Z,A) more accurate values have been determined. They have been tabulated by I. Angeli, K. P. Marinova, Atomic Data and Nuclear Data Tables \{\textbackslash{}bf 99\}, 69-\/95 (2013). The program {\ttfamily R\+N\+U\+C\+L\+E\+US} initializes a 2-\/dimensional array R\+R\+M\+S\+V\+E\+C(\+Z,\+A) and enters the non-\/zero values through assignment statements. The dimensions are R\+R\+M\+S\+V\+E\+C(120,300). Hence the program is exceedingly long. At the same time, the maximum number of mass values for a given Z is less than 30. A possible data type for each Z might be Type A\+\_\+rrms Integer \+:: a\+\_\+min Real(Kind=dp), Dimension(1\+:30) \+:: Z\+\_\+rrms End Type A\+\_\+rrms Where a\+\_\+min is the minimum value of A for the list of A-\/values. Then the list for the n\+\_\+atoms elements of the periodic table could be\+:

Type(atomic\+\_\+rrms), dimension(n\+\_\+atoms), parameter, public \+:: atoms\+\_\+rrms(/ \& A\+\_\+rrms( 1, Z\+\_\+rrms(1\+:3)=(/0.8783\+\_\+dp, 2.\+1421\+\_\+dp, 1.\+6591\+\_\+dp/) ), \& A\+\_\+rrms( 3, Z\+\_\+rrms(1\+:8)=(/1.9661\+\_\+dp, 1.\+6755\+\_\+dp, 0.\+0000\+\_\+dp, 2.\+0660\+\_\+dp, 0.\+0000\+\_\+dp, \& 1.\+9239\+\_\+dp/) ), \& Etc. Here we assume that Kind= dp has been declared to be a Real double precision (64-\/bit) value. The above needs to be tested.

The goal of this project is to derive a data type, like the above, for each atomic number Z and develop a module to initialize an {\ttfamily atoms\+\_\+rrms} array, and assign values through a data statements for the data type. This program could then replace the {\ttfamily geniso.\+f90} program in G\+R\+A\+SP. The easiest way might be to write a program that transforms the numbers in the assignment statements to data statements for the appropriate data type. 